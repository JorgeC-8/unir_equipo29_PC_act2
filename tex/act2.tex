\documentclass[11pt,a4paper]{markdown}
%----------------------------------------------------------------------------------------
% INFORMACIÓN DEL ARTÍCULO Y METADATOS
%----------------------------------------------------------------------------------------
\author{González Trillo Rodolfo Arturo}
\course{Percepción computacional}
\activity{Análisis y caracterización de las diferencias entre el sistema cognitivo humano y los sistemas artificiales en una función cognitiva concreta}
\assignment{2}
\title{Título}
\keywords{Inteligencia Artificial}
\date{\today}
%----------------------------------------------------------------------------------------


%%%%%%%%%%%%%%%%%%%%%%%%%%%%%%%%%%%%%%%%%%%%%%

\begin{document}

%%%%%%%%%%%%%%PÁGINA DE TÍTULO%%%%%%%%%%%%%%%%%%%%%%%
    \maketitle

    \begin{abstract}
       \noindent \textit{
        La revolución de la Inteligencia Artificial no sólo está ocurriendo en ámbitos industriales o empresariales, también se está extendiendo al ámbito educativo. Un problema aciago de la educación, es el hecho de que cada estudiante es distinto y sin embargo la clase es la misma para todos los que acuden a ella, sin importar sus capacidades. En la actualidad, este problema se ha paliado incluyendo una mayor diversidad de explicaciones y formas de aprendizaje en el aula, pero el éxito de estos enfoques es necesariamente limitado, pues aún subyace el problema de tener a el aula aprendiendo de la misma forma. 
        }
   \end{abstract}
  \vspace{1\baselineskip}
  \rule{\textwidth}{2pt}\vspace{2pt}
%%%%%%%%%%%%%%%%%%%%%%%%%%%%%%%%%%%%%%%%%%%%%

\indexed{section}{Introducción al contexto de negocio en el que se desarrolla el caso de uso}

\quad Estamos viviendo una revolución, una revolución tan profunda que está cambiando todos los aspectos de nuestras vidas: la revolución de la inteligencia artificial. Entre las muchas áreas que se están transformado vertiginosamente se encuentra el sector educativo. Pronto la imagen del profesor dictando dentro de un aula y docenas de alumnos en sus pupitres dormitando de aburrimiento, pertenecerá a los libros de historia. La revolución promete una educación más individualizada, efectiva, en mejora continua y más relevante. 

\quad Esta propuesta se centra en abordar un problema insidioso, que ha estado en la educación desde que se formó el sistema actual de aulas, profesores y alumnos, y —que a pesar de que se han hecho esfuerzos colosales con resultados muy satisfactorios— aún queda mucho por trabajar, puesto que el problema es la raíz misma de nuestro actual modelo educativo: el hecho de que cada estudiante es distinto, tiene necesidades diferentes e inteligencia distinta, pero aún se enseña a todos los estudiantes con la misma metodología: mismo profesor, mismo clase. La inteligencia artificial, combinada con un esquema híbrido de clases virtuales y clases en el aula, tiene el potencial de resolver este problema: si se tiene una plataforma con una enorme diversidad de contenidos educativos, se puede crear un sistema de recomendaciones automático, que aprenda de cada alumno y le recomiende contenidos educativos de acuerdo a sus preferencias y formas de estudiar, de manera que los resultados que obtenga se optimicen con el tiempo.

\quad Pero antes de describir a detalle nuestra propuesta, primero tenemos que entender de forma breve como se ha diferenciado las formas de aprender de cada estudiante, y además, el panorama actual de las empresas que se dedican a esta área emergente.

\begin{tcolorbox}[breakable, size=fbox, boxrule=1pt, pad at break*=1mm,colback=cellbackground, colframe=cellborder]
\prompt{In}{incolor}{1}{\boxspacing}
\begin{Verbatim}[commandchars=\\\{\}, fontsize=\scriptsize]
\PY{o}{\PYZpc{}}\PY{k}{matplotlib} inline
\PY{o}{\PYZpc{}}\PY{k}{config} InlineBackend.figure\PYZus{}format = \PYZsq{}png\PYZsq{}

\PY{k+kn}{import} \PY{n+nn}{numpy} \PY{k}{as} \PY{n+nn}{np}
\PY{k+kn}{import} \PY{n+nn}{matplotlib}\PY{n+nn}{.}\PY{n+nn}{pyplot} \PY{k}{as} \PY{n+nn}{plt}
\PY{k+kn}{import} \PY{n+nn}{math} \PY{k}{as} \PY{n+nn}{math}
\PY{k+kn}{from}  \PY{n+nn}{PIL} \PY{k+kn}{import} \PY{n}{Image}
\PY{k+kn}{from} \PY{n+nn}{tqdm}\PY{n+nn}{.}\PY{n+nn}{notebook} \PY{k+kn}{import} \PY{n}{tqdm}
\end{Verbatim}
\end{tcolorbox}

\begin{figure*}
\centering
\includegraphics[width=0.7\textwidth]{figures/logo.png}
\decoRule
\caption[Inteligencias múltiples de Gadner]{Esquema de la teoría de las inteligencias múltiples de. }
\fuente{Adaptada de.}
\label{fig:8-inteligencias-multiples}
\end{figure*}


\addcontentsline{toc}{section}{Referencias}
\printbibliography

\end{document}