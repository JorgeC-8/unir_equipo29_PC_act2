\documentclass[11pt,a4paper]{markdown}
%----------------------------------------------------------------------------------------
% INFORMACIÓN DEL ARTÍCULO Y METADATOS
%----------------------------------------------------------------------------------------
\author{ Rodolfo González, Jorge Castaneda, Carlos Alfaro}
\course{Percepción computacional}
\activity{Uso de filtros espaciales y morfológicos}
\assignment{2}
\title{Título}
\keywords{Percepcion Computacional}
\date{\today}
%----------------------------------------------------------------------------------------


%%%%%%%%%%%%%%%%%%%%%%%%%%%%%%%%%%%%%%%%%%%%%%

\begin{document}

%%%%%%%%%%%%%%PÁGINA DE TÍTULO%%%%%%%%%%%%%%%%%%%%%%%
    \maketitle

   
%%%%%%%%%%%%%%%%%%%%%%%%%%%%%%%%%%%%%%%%%%%%%

\indexed{section}{Conteo de palabras y caracteres en el ámbito académico}

\quad Hoy en día, el conteo de palabras y caracteres es un proceso muy utilizado e importante en el ámbito académico, ya que les otorga, tanto a estudiantes y docentes, una herramienta útil en la revisión de trabajos escritos de los cuales solo contamos con imágenes del documento. 

\quad En tiempos de pandemia la entrega de tareas o ensayos por fotografías incremento drásticamente, en su mayoría en educación a nivel primario y básico, y en sectores menos privilegiados en donde no cuentan con una red estable con conexión a internet o una computadora para uso personal del estudiante, donde la única comunicación con la que se cuenta con los docentes académicos y el estudiante era el teléfono y las redes sociales, por lo que las entregas de muchos ensayos eran mediante fotografías del documento escrito a mano, lo que complicaba demasiado al profesor para poder realizar su labor de revisión.

\quad Debido al incremento de incidencia en este problema, se decide implementar un sistema utilizando filtros espaciales y morfológicos que ayude al sector académico a contar las palabras y caracteres que tiene una imagen, junto con estadísticas extras como tamaño promedio de las palabras, posicionamiento de palabras más largas y guardado de palabras al azar, con el fin de facilitar la revisión y calificación de este.

\indexed{section}{Objetivos:}

\begin{itemize}
  \item Facilitar el proceso de calificación al docente académico, enfocado en sectores con estudiantes con problemas de conexión a internet y acceso a computadora en tiempos de pandemia.
  \item Ayudar al estudiante a tener una calificación más exacta y rápida sobre su ensayo sin tener la limitante de entregar el trabajo exclusivamente en un documento.
\end{itemize}

\indexed{section}{Ventajas:}

\begin{itemize}
  \item El sistema le aplicará distintos filtros a la fotografía, lo que ayudará a clarificar el ensayo del estudiante para que el profesor pueda revisarlo con menor esfuerzo.
  \item Normalmente los ensayos o trabajos a nivel académico básico cuentan con una cantidad mínima o máxima de palabras, el sistema ayudará al docente a no tener que contar las palabras de forma manual.
  \item El sistema obtendrá de estas imágenes datos que no son fácil de calcular para el docente, como el tamaño promedio de las palabras y posicionamiento de palabras más largas.
  \item El sistema guardará las imágenes de palabras tomadas al azar, con el fin de que el profesor pueda tomar una idea de lo que trata el ensayo sin necesidad de leer todo el ensayo.
\end{itemize}

\indexed{section}{Inconvenientes:}

\begin{itemize}
  \item El sistema al guardar imágenes de las palabras tomadas al azar, pueden existir escenarios donde se tomen palabras que no ayuden a la comprensión del ensayo.
  \item El sistema puede presentar inconvenientes si la imagen no cuenta con la cantidad de luz suficiente o el tamaño de la letra del ensayo es demasiado pequeño.
\end{itemize}



\end{document}
